\documentclass{report}

\usepackage{indentfirst}
\title{Rapport du projet 2 : ULBMP}
\author{LI Min-Tchun}
\date{12/05/2024}


\begin{document}
	\maketitle
	\tableofcontents

\chapter{Introduction}
Pour le deuxième projet du cours INFO-F-106, nous avons dû implémenter un compresseur d'image appelé ULBMP, basé sur le format BMP, développé par Microsoft dans les années 80.

Afin de pouvoir venir à bout de ce projet, il est demandé d'importer le module \textbf{Pyside6} afin de pouvoir créer un interface grahpique pour l'utilisateur. Il est donc important de lire la documentation de ce module afin de pouvoir l'utiliser correctement.

De plus, il est fondamental de maitriser la manipulation de fichier, bits (notamment le \textit{shifting} et le \textit{masking}), ainsi que la compréhension de la représentation binaire et hexadécimal.

Concernant la représentation binaire et hexadécimal, il était recommandé d'utiliser le programme \textbf{hexdump} à exécuter dans le terminal afin de pouvoir visualiser les données hexadécimale.
\chapter{ULBMP 1.0}
Ceci est le chapitre 1.

\chapter{ULBMP 2.0}
Ceci est le chapitre 2.

\chapter{ULBMP 3.0}
Ceci est le chapitre 3.

\chapter{ULBMP 4.0}
Ceci est le chapitre 4.

\chapter{Conclusion}
Ceci est une conclusion.

\end{document}
