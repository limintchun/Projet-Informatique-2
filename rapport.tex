\documentclass{report}

\usepackage{graphicx}
\usepackage{indentfirst}
\usepackage{listings}

\title{Rapport du projet 2 : ULBMP}
\author{LI Min-Tchun}
\date{12/05/2024}


\begin{document}
	\maketitle
	\tableofcontents

\chapter{Introduction}
Pour le deuxième projet du cours INFO-F-106, nous avons dû implémenter un compresseur d'image appelé ULBMP, basé sur le format BMP, développé par Microsoft dans les années 80.

Afin de pouvoir venir à bout de ce projet, il est demandé d'importer le module \textit{Pyside6} afin de pouvoir créer un interface grahpique pour l'utilisateur. Il est donc important de lire la documentation de ce module afin de pouvoir l'utiliser correctement.

De plus, il est fondamental de maitriser la manipulation de fichier, de bits (notamment le shifting et le masking), ainsi que la compréhension de la représen-tation binaire et hexadécimal.

En effet, comme nous devons utiliser des pixels, il est important de comprendre que ces derniers sont composées de 3 valeurs : Rouge, Vert, Bleu (RVB ou RGB en anglais). Chacune de ces valeurs est codée sur 8 bits, ce qui signifie que chaque valeur peut prendre $2^8 = 256$ valeurs différentes.


Il était donc utile d'utiliser le programme \textbf{hexdump}, à exécuter dans le terminal afin de pouvoir visualiser les données hexadécimale.

Cependant, une question se pose : comment peut-on représenter une image en format numérique ou bien comment peut-on compresser une image ? Afin de résoudre ce problème, on va procéder comme suit, on va créer une classe \textit{Decoder} et \textit{Encoder}. 

Le premier permettra, à partir d'un chemain d'accès du fichier, de lire chaque bytes et de retourner une \textit{image} (on définiera plus tard plus précisemment ce que c'est).  

Le deuxième permettra, à partir d'une \textit{image}, d'écrire dans un fichier les bytes correspondants aux valeurs RGB de chaque pixel.

Il est important de notifier qu'il y a plusieurs version du compresseur d'image. Cela signifie que le header (ce qui se trouve à chaque début de fichier pour reconnaitre le format du fichier) doit être adapté à chaque version.


\chapter{Initialisation}
Comme cité dans l'introduction, les classes \textit{Decoder} et \textit{Encoder} nécessiteent une image, définie par la classe \textit{Image}. Cette classe contient les attributs suivants : la largeur, la hauteur de l'image et les pixels qui la constitue. Ces derniers sont également définis à partir d'une classe, \textit{Pixel} qui contient les attributs suivants : la valeur de rouge, de vert et de bleu. 

Bien évidemment, certaines fonctions spécifiques à la construction de classes doivent être créé pour faciliter la manipulation des pixels et des images. Par exemple, la fonction __getitem__ permet de récupérer un pixel à partir de ses coordonnées (x, y).
\begin{lstlisting}
def __getitem__(self, pos: tuple[int, int]):
        """
        Surcharge l'opérateur [] en lecture
        Lance une erreur si pos n'est pas une position valide dans l'image.
        """
        x = pos[0]
        y = pos[1]
        if not (x <= self.w and y <= self.h):
            raise IndexError("Il n'y a pas assez de pixels pour remplir l'image.")
        return self.img[y * self.w + x]

\end{lstlisting}



\chapter{ULBMP 2.0}
Ceci est le chapitre 2.

\chapter{ULBMP 3.0}
Ceci est le chapitre 3.

\chapter{ULBMP 4.0}
Ceci est le chapitre 4.

\chapter{Conclusion}
Ceci est une conclusion.

\end{document}
